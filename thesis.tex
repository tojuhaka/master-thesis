\documentclass[utf8]{gradu3}

\usepackage{graphicx} % kuvien mukaan ottamista varten

\usepackage{amsmath} % hyödyllinen jos tekstisi sisältää matikkaa,
                     % ei pakollinen

\usepackage{booktabs} % hyvä kauniiden taulukoiden tekemiseen
\usepackage{listings}
\usepackage{color}
\usepackage{qtree}

% HUOM! Tämän tulee olla viimeinen \usepackage koko dokumentissa!
\usepackage[bookmarksopen,bookmarksnumbered,linktocpage]{hyperref}

\addbibresource{thesis.bib} % Lähdetietokannan tiedostonimi

\definecolor{light-gray}{gray}{0.96}

\lstdefinelanguage{Smalltalk}{
morekeywords={true,false,self,super,nil},
sensitive=true,
morecomment=[s][\color{blue}]{"}{"},
morestring=[d]',
alsoother={_},
style=SmalltalkStyle,
columns=fullflexible,
backgroundcolor=\color{light-gray},
xleftmargin=.20in,
showspaces=false,
numbers=left,
framexleftmargin=15pt,
tabsize=4
}
\lstdefinestyle{SmalltalkStyle}{
literate={:=}{{$\gets\ $}}2{^}{{$\uparrow$}}1{_}{{$\gets\ $}}2{ä}{{\"a}}2{ö}{{\"o}}2
}


\begin{document}

\title{MVC-arkkitehtuurin toteutus web-sovelluskehyksissä}
\translatedtitle{Usage of the {gradu3} document class for \LaTeX\ theses}
\studyline{Kaikki suuntautumisvaihtoehdot}
\avainsanat{%
  \LaTeX,
  {gradu3},
  pro gradu -tutkielmat,
  kandidaatintutkielmat,
  käyttöohje}
\keywords{\LaTeX, {gradu3}, Master's Theses, Bachelor's Theses, user's guide}
\tiivistelma{%
Tässä tutkielmassa esitellään MVC-arkkitehtuurin toteutusta Python-pohjaisissa web-sovelluskehyksissä. Työssä selvitetään
millä tavalla MVC on toteutettu sovelluskehyksissä ja vastaako se alkuperäistä MVC:n toteutusta, joka määritellään Krasnerin artikkelissa \parencite{krasner}. Työssä tutkitut 
sovelluskehykset ovat Django, Pyramid ja Flask. Sovelluskehykset eivät toteuttaneet MVC:tä niinkuin se on määritelty. Flaskin avulla ei myöskään voida toteuttaa MVC:tä Krasnerin vaatimusten mukaisesti.

}
\abstract{%
  This document is a sample {gradu3} thesis document class
  document.  It also functions as a user manual and supplies
  guidelines for structuring a thesis document.

  The English abstract of a thesis should usually say exactly the same
  things as the Finnish tiivistelmä.
}

\author{Toni Haka-Risku}
\contactinformation{Ag~C416.1, \texttt{toni.haka-risku@student.jyu.fii}}
% jos useita tekijöitä, anna useampi \author-komento
\supervisor{Ohjaamaton työ}
% jos useita ohjaajia, anna useampi \supervisor-komento

\maketitle

\preface
Tutkielman aiheen valintaan vaikutti aiempi kokemus ohjelmoinnista niin vapaa-ajalla kuin työelämässäkin. Suurimmassa osassa ohjelmistoprojekteja, joissa olen työskennellyt, on ollut käytössä MVC-pohjainen sovelluskehys työkaluna. Django-sovelluskehyksen sivuilla kuitenkin kyseenalaistetaan MVC:n toteutus sellaisenaan. Tästä sain idean kirjoittaa Pro Gradu tutkielman löytääkseni vastauksen kyseiseen ongelmaan. 
Jyväskylässä \today

\bigskip

Tutkielman tekijä


\begin{thetermlist}
\item[Sovellusarkkitehtuuri] kertoo millä tavalla sovellus on rakennettu
\item[MVC] Model-View-Controller sovellusarkkitehtuuri
\item[Sovelluskehys] on sovelluskehitykselle tarkoitettu pohja, joka kutsuu käyttäjän kirjoittamaa ohjelmakoodia ja tarjoaa ratkaisuja yleisimpiin toistuviin ongelmiin
\item[WSGI] määrittää miten web-palvelin kommunikoi web-sovellusten kanssa ja millä tavalla web-sovellukset
yhdistetään prosessoimaan pyyntöjä.
\item[Template] Dynaaminen html-pohja, johon voidaan kirjoittaa ohjelmalogiikkaa
\item[Piirikytkentä] Dynaaminen html-pohja, johon voidaan kirjoittaa ohjelmalogiikkaa
\item[Pakettikytkentä] on tiedonsiirtomenetelmä, jossa yhteys data jaetaan paketeiksi. Osapuolien väliseen viestintään ei varata kaistaa, eikä osapuolien välille luoda katkeamatonta yhteyttä \parencite{kytkenta}.
\item[Piirikytkentä] on tiedonsiirtomenetelmä, jossa yhteys pysyy auki riippumatta siitä, onko osapuolien välillä liikennettä. Piirikytkentä on käytössä esim. puhelinverkoissa \parencite{kytkenta}.
\item[Asiakas-sovellus] alustaa yhteyden lähettääkseen pyyntöjä palvelimelle \cite{http}.
\item[Käyttäjä-agentti] on tietokoneohjelma, joka lähettäää pyynnön jollekin tietylle ohjelmalle. Tällaisia ovat esimerkiksi selaimet \cite{http}. 
\item[Palvelin] on tietokoneohjelma, joka ottaa vastaan yhteyksiä ja lähettää takaisin vastauksia asiakas-sovellukselle. Jokainen ohjelma pystyy olemaan yhtäaikaa asiakas-sovellus ja palvelin \cite{http}. 
\end{thetermlist}

\mainmatter

\chapter{Johdanto}
\section{Tutkimuskysymys}
MVC-arkkitehtuuri on saanut paljon huomiota web-sovelluskehyksien
toteutuksissa ja useat web-sovelluskehykset ovat luokiteltu
MVC-pohjaisiksi sovelluskehyksiksi \parencite{mvcframeworks}. Se on kuitenkin alunperin
tarkoitettu matalan tason käyttöliittymäsovellusten toteuttamiseen,
jossa esimerkiksi hallitaan yksittäisiä näppäimistöltä tulleita
syötteitä eikä sitä ole suoraan tarkoitettu käytettäväksi
web-sovellusten ohjelmointiin. Alkuperäisen MVC:n toteutuksen
soveltuvuutta web-ohjelmointiin onkin epäilty. Esimerkiksi Leff
soveltaa artikkelissaan MVC:n käyttämistä web-sovelluksissa, mutta
samalla esittelee alkuperäisen MVC:n toteuttamisen ongelmana. Tämä johtuu web-sovelluksen jakautumisesta asiakkaan
(client) ja palvelimen (server) välille \parencite{ibm_watson}. Myös Pyramid-sovelluskehyksen tekijät
kyseenalaistavat MVC-arkkitehtuurin toteutuksen Pyramidissa ja uskovat
MVC:n olevan sellaisenaan sopimaton web-ohjelmointiin, vaikka
Pyramidin toteutus onkin hyvin lähellä alkuperäistä MVC:tä
\parencite{pyramid}. Django on vastaavasti toteutettu MVC:n pohjalta, mutta sen väitetään myös toteuttavan MVC hieman erilailla kuin MVC on alunperin tarkoitettu \parencite{django_mvc}.

Tutkimuksen tarkoituksena on selvittää ohjelmistokehityksen ja ohjelmistoarkkitehtuurejen taustoja. Näiden pohjalta siirrytään tutkimaan millä tavalla 
MVC-arkkitehtuuri on toteutettu web-sovelluskehyksissä ja miten se
eroaa alkuperäisestä MVC:n toteutuksesta \parencite{krasner}. Lisäksi pyritään toteuttamaan MVC alkuperäisen
toteutuksen mukaan käyttäen Flask-sovelluskehystä. Havaintojen pohjalta
selvitetään voidaanko MVC tuoda sellaisenaan web-sovelluskehykseen.

\chapter{Tutkimuksen Rakenne}
Tutkimus aloitetaan kirjallisuuskatsauksella, jossa tarkastellaan mitä aiempaa 
tutkimusta ohjelmistokehityksestä ja MVC:stä on tehty. Lisäksi käydään
läpi mitä lähteitä löytyy Python-pohjaisista web-sovelluskehyksistä sekä ohjelmistoarkkitehtuureista. Tämän jälkeen 
tutkitaan MVC:n historiaa sekä millä tavalla MVC on tarkoitettu toteutettavaksi.
Tässä vaiheessa käydään läpi jokaisen MVC-komponentin tarkoitus sekä niiden 
keskinäisen kommunikaation rakentuminen. Lisäksi esitellään
Dortmundin yliopistossa kirjoitettu esimerkkiohjelma Smalltalkilla siitä miten MVC:n 
toteutus tuodaan sovellukseen käytännössä.

MVC:n tarkastelun jälkeen esitellään tutkimuksessa käytetyt web-sovelluskehykset, 
joita käytetään apuna MVC:n tutkimisessa. Sovelluskehyksistä käydään läpi sen 
historia sekä yleisellä tasolla mihin käyttötarkoitukseen
sovelluskehys on tarkoitettu. Tämän jälkeen verrataan MVC:n toteutusta erikseen 
jokaiseen sovelluskehykseen ja selvitetään millä tavalla niiden sovellusarkkitehtuuri 
mahdollisesti eroaa MVC:stä,
Havaintojen perusteella pohditaan MVC:n mahdollisia ongelmia sovelluskehyksien toteutuksessa 
ja selvitetään löytyykö sovelluskehyksien arkkitehtuurista jotain yhtenäisiä piirteitä, mitkä ovat
kytköksissä MVC:n toteutukseen. Saatujen tulosten pohjalta  kirjoitetaan Flask-sovellus, joka toteuttaa MVC:n 
niinkuin se on alunperin tarkoitettu.

Tutkimuksen lopuksi koostetaan havainnoista yhteenveto, jossa pohditaan saatuja tuloksia ja selvitetään 
pystytäänkö niiden perusteella vastaamaan tutkimuskysymykseen.


\chapter{Aiheen rajaus}
Aluksi tarkastellaan ohjelmistokehitystä ja ohjelmistoarkkitehtuureja yleisesti pohjustamaan MVC-arkkitehtuuria. MVC:n toteutus käydään
hyvin yksityiskohtaisesti läpi ja sen vertaaminen rajataan Python-websovelluskehyksiin. Tarkasteltavat web-sovelluskehykset rajataan 
Pyramid-, Django- sekä Flask-sovelluskehyksiin. Pyramid ja Django toteuttavat MVC:n kaltaisen
sovellusarkkitehtuurin. Flask on sovelluskehys, joka tarjoaa vain
välttämättömät kirjastot web-sovelluksen toteuttamiseen. Sitä
käytetään tutkimuksessa työkaluna selvittämään miten MVC tulisi
toteuttaa sovelluskehykseen niinkuin se on tarkoitettu. MVC:stä on olemassa erilaisia versioita, joten sen määrittely tulee rajata tarkasti.
Kun puhutaan MVC:stä tarkoitetaan tällä Krasnerin
artikkelissa esiteltyjä määrittelyitä MVC:n
toteutuksesta \parencite{krasner}, jotka pohjautuvat Trygve Reenskaugin esittelemään 
MVC:n määritelmään \parencite{xerox-original}. 

Tarkasteltavat sovelluskehykset käydään ensiksi yleisellä tasolla läpi, jonka jälkeen niitä tarkastellaan MVC:n näkökulmasta. Yleisellä tasolla tarkoitetaan sovelluskehyksen historian ja käyttötarkoituksen esittelemistä. Sovelluskehyksien muihin teknisiin ominaisuuksiin ei oteta kantaa. Flask-osiossa MVC-arkkitehtuuri toteutaan niinkuin se on Krasnerin julkaisussa määritelty. Näiden havaintojen pohjalta pyritään vastamaan tutkimuskysymykseen.


\chapter{Kirjallisuuskatsaus}

Kirjallisuuskatsauksessa käydään läpi vaihe vaiheelta, miten
lähdemateriaalia kerätään tutkimusta varten. Lähdemateriaalin 
haku toteutetaan hakukoneilla, jotka ovat tarkoitettu erityisesti 
tieteellisten artikkeleiden etsimiseen. Tässä tutkielmassa käytetyt 
hakukoneet ovat seuraavat: IEEE Xplore, ACM Digital Library, 
Google Scholar sekä joissakin tapauksissa Google:n yleinen
hakukone. Yleistä hakukonetta on käytetty esimerkiksi
sovelluskehyksien dokumentaatioiden etsintään.

Aluksi muodostetaan kokonaiskuva tuloksista, jolloin silmäillään läpi
saatuja artikkeleita. Tässä vaiheessa tarkoitus ei ole vielä valita
mitään pohjaksi tutkielmalle, vaan kerätä informaatiota siitä
millainen lähdemateriaali on tarjolla kokonaisuudessaan. Saaduista 
tuloksista poimitaan artikkeleita, jotka sopivat tutkimuksen aihepiiriin.
Seuraavaksi artikkeleista valitaan tutkielmalle
pohjakirjallisuus. Tässä vaiheessa artikkelit luetaan huolellisesti
läpi ja varmistutaan siitä, että ne ovat tieteellisesti päteviä
tutkielmaa varten. Erityisesti kiinnitetään huomiota viittauksien
määrän valittaessa tärkeimmät lähdemateriaalit. Tutkielmassa esiintyy myös satunnaisia viittauksia,
joita ei ole kirjallisuuskatsauksessa mainittu. Tutkimuksen
pääkirjallisuus kuitenkin käydään läpi kirjallisuuskatsauksessa.
Haussa käytetään seuraavia hakutermejä: "Software engineering", "Software development", "MVC", "MVC Architecture",
"frameworks", "web frameworks" ja  "MVC- Architecture".
Tarkasteltavat artikkelit rajataan kuitenkin niihin,
jotka esittelevät suoraan MVC:tä itseään, tarjoavat
lähdemateriaalin sovelluskehyksien esittelyyn tai auttavat pohjustamaan yleisesti ohjelmistokehitysten tarkastelua.

\section{Sovelluskehitys ja arkkitehtuurit}
Sovelluskehityksen tarkasteluun löytyy runsaasti lähdemateriaalia. Merkittäväksi lähteeksi valitaan Ian Sommervillen kirjoittama \textit{Software Engineering 9th Edition} -kirja, jossa käydään läpi mitä kaikkea isojen ohjelmistojen kehitykseen kuuluu \parencite{Sommerville}. Kirja on jaettu neljään pääosioon: \textit{Introduction to Software Engineering}, \textit{Dependability and Security}, \textit{Advanced Software Engineering} ja \textit{Software Management}. Tutkimuksessa lähteenä käytetään \textit{Introduction to Software Engineering} ja \textit{Advanced Software Engineering} -osioita. \textit{Introduction to Software engineering} käy läpi ohjelmistojen suunnittelua ja toteutusta yleisesti. Tällaisia ovat esimerkiksi vaatimusmäärittelyt, prosessit, arkkitehtuuri ja testaus. \textit{Advanced Software Engineering} esittelee ohjelmistojen kirjoittamista siten, että ne olisivat mahdollisimman uudelleen käytettäviä ja ylläpidettäviä. Tähän tarjotaan ratkaisuksi erilaisia sovellusarkkitehtuureja (esim.  \textit{Component-based architecture} ja \textit{Service-oriented architecture} \parencite{Sommerville}.



\section{MVC}
Google Scholarin tuloksista löytyy kolme artikkelia MVC:stä, jotka sopivat 
lähdemateriaaliksi tutkimukseen. Ensimmäinen artikkeleista on John
Deaconin  kirjoittama artikkeli, joka tarkastelee lyhyesti
MVC:tä \parencite{deacon}. Artikkeli on kuitenkin hyvin suppea, mutta selittää
tiivistetysti MVC:n idean. Toinen artikkeli on Steve burbeckin kirjoittama, 
joka käsittelee
MVC:tä sellaisena kuin sitä käytettiin Smalltalkissa \parencite{burbeck}. burbeckin
artikkeliin viitataan monissa MVC:tä käsittelevissä
julkaisuissa, joten sen arvo tämän tutkielman pohjakirjallisuudessa on
vahva. Viittausten määrä on katsottu hakemalla artikkelia Google
Scholarin hakukoneessa. Seuraavaksi kartoitetaan pohjakirjallisuutta käyttäen ACM Digital
Library sekä IEEE XPlore -hakukoneita. Kolmas artikkeli Glenn
E. Krasnerin kirjoittama julkaisu, jossa esitellään MVC:n toteutusta
erilaisissa Smalltalk-sovelluksissa. Julkaisusta löytyy useita
versioita, joista tässä tutkielmassa käytetään
molempia \parencite{krasner} \parencite[krasner_desc]. Tähän artikkeliin on
myös viitattu runsaasti, joten se on burbeckin julkaisun kanssa
tärkeimpiä lähteitä MVC:n pohjakirjallisuudessa. Kirjoitushetkellä
viittauksia Krasnerin artikkeliin on 2263. Monien MVC-arkkitehtuuria soveltavien artikkeleiden
lähdeviitteistä löytyy viittauksia burbeckin ja Krasnerin
artikkeleihin. Tämän perusteella pystytään toteamaan kyseisten
artikkeleiden olevan tieteellisesti päteviä ja tarjoavan kattavan
lähdemateriaalin MVC:n pohjaksi. burbeckin ja Krasnerin kirjoittamien
artikkeleiden taustalta löytyy MVC-arkkitehtuurin alkuperäinen kehittäjä Trygve
Reenskaug, jonka omia julkaisuja sekä kotisivujen MVC-osiota käytetään
myös lähteenä tutkielmassa \parencite{xerox}. Erityisesti Reenskaugin ja Adele Goldbergin
julkaisu, jossa kerrotaan jokaisen MVC komponentin tehtävä....

\section{Web-sovelluskehykset \& Internet}
Internetin taustojen selvitykseen käytetään IETF:n (The Internet Engineering Task Force) dokumentaatioita \parencite{ietf}. Näistä erityisesti TCP:n ja Iternet Protokollan dokumentaatiot Web-sovelluskehyksistä löydetty kirjallisuus on hyvin
suppea, eikä tämän varaan voida rakentaa kovinkaan perusteellista
tieteellistä pohjaa. Tämän vuoksi tutkimuksessa joudutaan osaksi turvautumaan
sovelluskehyksien omaan dokumentatioon täydentämään
lähdemateriaalia. IEEE Xploren ja ACM Digital Libraryn avulla löytyy kolme julkaisua, 
joita käytetään tutkimuksen pohjana sovelluskehyksiä
tarkastellessa. Ensimmäinen artikkeli on Okanovicin ja Mateljan
kirjoittama artikkeli, jossa esitellään web-sovelluskehyksien suunnittelua \parencite{ockanovic}. 
Se myös sivuuttaa lyhyesti MVC:tä. Toisena artikkelina käytetään ACM:stä tuloksena saatua Iwan
Vosloon julkaisua, jossa käydään läpi yleisesti web-sovelluskehyksien
rakennetta \parencite{vosloo}. Kolmanneksi hyödynnetään Ignacion artikkelia, jossa esitellään ketteriä
web-sovelluskehyksiä sekä millä tavalla näitä tulisi vertailla \parencite{ignacio}. Lisäksi käytetään
IEEE:stä Ahamedin julkaisua, joka esittelee yleisesti asioita joita tulisi ottaa huomioon sovelluskehystä 
valittaessa \parencite{towards_framework}.

Google Scholarin hakutuloksista löytyi Liza Daly:n
kirjoittama ja O'Reillyn julkaisema "Next Generation Web Frameworks
in Python", joka sisältönsä puolesta sopii hyvin pohjaksi
tutkimuksessa käsiteltävien sovelluskehyksien lähdemateriaaliksi 
\parencite{liza}.


\chapter{Internet}
Internet ja sen arkkitehtuuri on syntynyt hyvin pienistä saavutuksista. Internetin taustalla ei ole alunperin ollut minkäänlaista suurempaa suunnitelmaa. Tutkiessamme Iternetin arkkitehtuuria tulee muistaa, että tekniset muutokset ovat jatkuvia tietotekniikan alalla. Monet arkkitehtuurilliset ratkaisut, jotka olivat suosiossa muutamia vuosia sitten ovat nyt jo vanhentuneet. Taas ratkaisut, jotka tänä päivänä nähdään hyvinä, voivat jo huomenna olla huonoja. Voidaan sanoa, että ainut asia joka Internetin suhteen pysyy on jatkuva muutos. Internet yhteisöjen monet jäsenet väittävät, että Internetillä ei ole arkkitehtuuria, mutta kuitenkin eräänlaisia perinteitä, johon koko Iternet pohjautuu. Iternetin tavoite on tarjota yhteydet, jossa työkaluna käytetään Iternet Protokollaa. Monien yhteisön jäsenien mielestä tärkeää ei ole se millä tavalla yhteydet luodaan verkon sisällä, vaan se millä tavalla yhteyksien alku- ja loppupää käyttäytyvät. Tähän tärkeimpänä argumenttina pidetään sitä, että useat tarvittavat toiminnallisuudet voidaan toteuttaa vain IP:tä käyttävien systeemien puolella. Jokainen hyvin suunniteltu verkko sisältää mahdollisuuden jollakin tasolla epäonnistua viestin kuljettamisessa. Tällaisen ongelman kanssa tulisi toimia siten, että vastuu kommunikaation eheydestä annetaan ulkopuolisille sovelluksille.  Nykyinen Internetin exponentaalinen kasvu näyttää kuitenkin sen, että yhteydet ovat paljon arvokkaampia kuin yksittäiset sovellukset. Tällaiset yhteydet vaativat yhteistyötä palveluntarjoajien ja televiestintäympäristöjen välille\cite{constant_change}. 

Yleisesti on haluttua, että olisi vain yksi protokolla, jonka kautta yhteydet voidaan muodostaa. Tämän protokollan päälle voidaan kuitenkin rakentaa monia muita protokollia, jotka hoitavat muita erilaisia tarpeita. Käytännössä on ainakin kaksi syytä miksi verkko-protokollien tasoja tarvitaan useita. Esimerkiksi eri IP:stä voidaan haluta vaihtaa toiseen versioon. Myös uudet vaatimukset voivat synnyttää uusia protokolia. Internet tason protokolla tulee olla täysin riippumaton laitteistosta. Tällöin Internet voidaan toteuttaa minkä tahansa digitaalisen teknologian päälle.
 \cite{constant_change}.

\section{Historia}
Internetin historia on lähtöisiin tarpeesta luoda yhteyksiä tietokoneiden välille, joiden avulla voidaan luoda verkkoja. Verkkojen kautta erilaiset sovellukset voivat keskustella keskenään. Tämä on johtanut tarpeeseen luoda standardit, jotka määrittävät sen millä tavalla yhteyksiä tulisi luoda ja ylläpitää. TCP (Transmission Control Protocol) on standardi, joka määrittää yhteyden luonnin tietokoneiden välillä sekä niiden välisen kommunikoinnin. TCP:n avulla sovellukset pystyvät lähettämään dataa toisilleen verkon yli. TCP kehitettiin Yhdysvaltain asevoimien tutkimusorganisaatiossa DARPA:ssa (Defense Advanced Research Projects Agency )\cite{tcp1_1}. 

\section{TCP}
TCP on tarkoitettu toimivan eri tasoihin jaetussa protokollien hierarkiassa, joka tukee verkko-sovelluksia. Sen tarkoitus on pystyä toimimaan monien alemman tason protokollien päällä ja olla ottamatta kantaa käytetäänkö yhteyden muodostuksessa esimerkiksi paketti- tai piirikytkentää. TCP tarjoaa luotettavan DARPA määrittää eri protokollien tasot seuraavasti:


\begin{figure}[h]
\centering
\includegraphics[scale=0.85]{protocol_layering.PNG}
\caption{\cite{tcp1_1}}
\end{figure}


TCP on rakennettu Internet Protokollan päälle, joka tarjoaa TCP:lle työkalut määrittää lähettäjän ja vastaanottajan osoitteet erilaisissa verkoissa. Internet Protokolla hoitaa myös mahdollisten TCP-segmenttejen pirstaloinnin ja uudelleenkokoamisen, jolloin TCP:n ei tarvitse näistä huolehtia. Se myös pitää mukanaan tiedon TCP-segmenttejen järjestyksestä, tietoturvamäärityksistä sekä lajittelusta. TCP tarjoaa rajapinnan luotettavalle yhteydelle useista verkoista koostuvassa ympäristössä \cite{tcp1_1} 

TCP koostuu kahdesta rajapinnasta, joista toinen toimii käyttäjälle tai sovellukselle ja toinen alemman tason protokollille kuten Internet Protokolalle. Sovelluksille tarkoitettu rajapinta koostuu vastaavista kutsuista mitä käyttöjärjestelmä tarjoaa tiedostojen hallintaan. Esimerkiksi se sisältää kutsut yhteyden avaamiseen ja sulkemiseen sekä datan vastaanottoon ja lähetykseen yhteyden läpi. TCP pystyy myös asynkronisesti kommunikoimaan sovelluksien kanssa. Rajapinta alemmille protokollille on yleisesti määrittelemätöln, mutta se olettaa rajapinnassa olevan mekansimin, jossa kaksi tasoa pystyvät asynkronisesti siirtämään dataa toistensa välillä. Yleensä kyseinen määrittely oletetaan löytyvän alemman tason protokollalta. TCP on tarkoitettu toimivan geneerisessä ympäristössä, mutta yleisesti voidaan olettaa alemman tason protokollan olevan Internet Protokolla \cite{tcp1_4}.



\section{TCP:n toiminta}
TCP:n tärkein tehtävä on tarjota turvallinen ja luotettava yhteys kahden prosessin välille. Tarjotakseen tällaisen palvelun Internetin päälle,tulee tietyt osa-alueet toteutua. Alla TCP:n toteuttamat osa-alueet listattuna selityksineen:

\begin{desclist}
\item[Datan siirto] TCP pystyy lähettämään jatkuvasti paketteja kummankin käytätjän suuntaan. Joskus käyttäjien tulee varmistaa, että kaikki lähetetty data on lähetetty vastaanottajalle. Tähän tarkoitukseen on kehitelty push -toiminto, joka käskee TCP:tä
lähettämään datan vastaanottajalle \cite{tcp1_4}.
\item[Luotettavuus] TCP:n tulee pystyä selviytymään tilanteesta, jossa data on jollain tavalla korruptoitunut tai hävinnyt. Tämä on ratkaistu lisäämällä jokaiseen pakettiin sekvenssinumero ja jokaisessa viestissä vaaditaan kuittausta vastaanottajalta. Jos kuittausta ei jostain syystä saada, lähetetään paketti uudelleen. Vastaanottajan puolella sekvenssinumeron avulla pystytään päättelemään oikea järjestys \cite{tcp1_4}.
\item[Datavirran hallinta] TCP antaa vastaanottajalle mahdollisuuden rajata datamääriä. Tämä mahdollistetaan antamalla vastaanottajalle mahdollisuus lähettää kuittausviestin mukana ikkuna, joka määrittää rajat datalle \cite{tcp1_4}.
\item[Kanavointi] Koska yhden isäntäkoneen on mahdollista pavella useita TCP-yhteyksiä rinnakkain, tulee TCP:n tarjota lista osoitteita ja portteja yksittäistä severiä kohden. Jokaiselle yhteydelle luodaan soketti (socket). Soketti-pari muodostaa uniikin yhteyden, missä yksittäistä sokettia voidaan myös käyttää moniin eri yhteyksiin rinnakkain. Hyvänä esimerkkinä yhden soketin jakamisesta on erilaiset logitusprosessit, joita käytetään jatkuvasti.  Erillisiin portteihin liittäminen hoidetaan aina jokaisen isäntäkoneen toimesta \cite{tcp1_4}.
\item[Yhteydet] Luotettavuuden ja kanavoinnin toteutus vaatii TCP:tä erillisen tilan alustamista ja ylläpitoa jokaiselle datavirralle. Tietoa soketeista, sekvenssinumeroista ja ikkunoiden kokoa kutsutaan yhteydeksi. Jokainen yhteys määritellään sokettiparin perusteella. Kun kaksi prosessia haluaa kommunikoida kekskenään, kummankin TCP:n tulee ensin luoda yhteys, jolloin kummankin tila alustetaan. Kun yhteys on valmis, se terminoidaan tai suljetaan resurssien vapauttamiseksi. Koska yhteydet luodaan epävakaan internet-yhteyden sekä epäluotettavien prosessejen välille, TCP käyttää aikapohjausta kättely-mekanismia yhteyden muodostamisessa. Tämän avulla vältetään virheelliset yhteydenmuodostukset \cite{tcp1_4}.
\item[Priorisointi ja tietoturva] TCP:n käyttäjät voivat hallita priorisointia sekä tietoturvaa erilaisilla arvoilla. Tällainen on esimerkiksi TCP-oktetin sisältämät ensimmäiset kolme bittiä, jotka määrittävät oktetin tärkeysjärjestyksen \cite{tcp_terminology}
\end{desclist}


TCP:n oletetaan olevan tavallinen moduuli käyttöjärjestelmässä. Käyttäjien tulee pystyä hallitsemaan TCP:tä niinkuin mitä tahansa tiedostojärjestelmää. Se pystyy myös kutsumaan muita käyttöjärjestelmän toimintoja. Rajapinta itse verkkoon tulee kuitenkin toteuttaa laitteen ajurissa. TCP ei kuitenkaan kutsu ajurin rajapintaa suoraan, vaan käyttää Internet Protokollaa. Internet protokolla sen sijaan kutsuu mahdollisen alemman tason ajurin rajapintaa. Käyttäjälle tarkoitettu rajapinta toteutetaan TCP:ssä siten, että sen kutsut ovat vastaavia kuin muissakin käyttöjärjestelmän ohjelmien tarjoamissa kutsuissa.  TCP:n tarjoamassa rajapinnassa kutsut ovat seuraavat:

\begin{desclist}
\item[OPEN] Avaa yhteyden
\item[CLOSE] Sulkee yhteyden
\item[SEND] Lähettää dataa
\item[RECEIVE] Ottaa vastaan dataa
\item[STATUS] Palauttaa yhteyden tilan
\end{desclist}

todo jatka tätä hieman

\cite{tcp2_3}

\section{Internet Protokolla}

Internet Protokolla on tarkoitettu käytettäväksi tietokoneiden välisiin verkkoihin, joissa on käytössä pakettikytkentä. Se tarjoaa toteutuksen siirtämään datagrammeja käyttäjien välillä. Se toteuttaa myös pitkien datagrammejen pirstaloinnin pienempiin osiin tarvittaessa, jos dataa halutaan siirtää verkkojen läpi, jossa vain pienemmät paketit ovat sallittuja. Internet Protokolla ei ota kantaa datan luotettavuuteen, datavirran hallintaan tai järjestykseen. Tämä jätetään ylemmän tason protokollien hoidettavaksi. Esimerkiksi TCP-moduuli kutsuu Internet moduulia ottamaan haltuun TCP-segmentin, joka sisältää otsikon ja lähettäjän datan. Otsikko sisältää osoitteen ja muut tarvittavat parametrit. Tämän jälkeen Internet-moduuli luo internet datagrammin ja kutsuu paikallista verkkorajapintaa lähettämään datagrammin eteenpäin. \cite{internet_protocol}

Internet Protokolla toteuttaa kaksi toiminnallisuutta: osituksen ja osoitteistuksen. Datagrammin otsikoissa tulleiden tietojen perusteella internet moduuli osaa lähettää datan oikeaan paikkaan sekä pirstaloida datan sopivaksi verkolle. Internet Protokolla käsittelee jokaista datagrammia erillään ilman riippuvuuksia muista datagrammeista. Internet Protokolla käyttää neljää palvelua: Palvelutyyppi, Elinkaari, Vaihtoehdot ja Tarkistussuma. Palvelutyyppi on abstrakti tai geneerinen lista parametrejä, jotka määrittää verkolle minkä tyyppinen palvelu on kyseessä. Yhdyskäytävät käyttävät tätä informaatiota valitsemaan oikeat parametrit tietylle verkolle tai määrittämään seuraavan verkon mihin hypätä. Yhdyskäytävillä tarkoitetaan tässä verkon solmua, joka mahdollistaa siitymisen toiseen verkkoon. Solmut voivat olla esimerkiksi tietokoneita. Elinkaarella tarkoitetaan ylärajaa, jonka aikana datagrammin tulee saavuttaa määränpäänsä. Jos määränpäätä ei saavuteta, datagrammi tuhotaan. Yläraja määritetään lähettäjän toimesta ja sitä vähennetään siirron aikana. Datagrammi tuhoutuu saavuttaessan arvon nolla \cite{internet_protocol}. Vaihtoehdot ovat käytössä erityisissä tilanteissa. Tällaisia ovat esimerkiksi tietyt ohjaukset verkossa tai lisäykset tietoturvaan. Tarkistussummalla varmistetaan, että datagrammi on siirretty onnistuneesti. Datagrammin sisältö voi kuitenkin sisältää virheitä. Jos Tarkistussummassa on virheitä, datagrammi tuhotaan välittömästi. 

Internet Protokolla ei tarjoa luotettavaa kommunikaatioyhteyttä. Se ei sisällä alku- ja loppupään kuittauksia eikä se ota kantaa datan eheyteen. Se ei myöskään sisällä uudelleenlähetyksiä tai muuta datavirran hallintaa \cite{internet_procotol}.
\section{HTTP-protokolla}
HTTP (Hypertext Transfer Protocol) on ollut Internetin WWW:n (World-Wide-Web) käytössä vuodesta 1990 lähtien. HTTP on sovellus-tason protokolla, jonka yhtenä tarkoituksena on siirtää hyperteksiä järjestelmien välillä. Hypertekstillä tarkoitetaan tietokoneissa käytettyä tekstiä, jossa mahdollistetaan ristiviittauksia dokumenttejen välillä. Näitä viittauksia kutsutaan hyperlinkeiksi \cite{hypertext}.   HTTP on geneerinen sekä tilaton ja sitä voidaan käyttää muuhunkin kuin hypertekstin siirtämiseen ylätunnisteiden, pyyntöjen sekä virhekoodejen kautta. Tällaisia ovat esimerkiksi nimipalvelumet ja jaetut oliopohjaiset hallintajärjestelmät. HTTP-protokolla mahdollistaa järjestelmien rakentamisen riippumattomana siirrettävästä datasta \cite{http}. HTTP tuo sovelluksille mahdollisuuden kommunikoida avoimella joukolla menetelmiä ja ylätunnisteita, jotka määrittävät itse HTTP-viestin tarkoituksen. HTTP käyttä URI:n spesifikaatiota määrittämään resurssejen sijainnin \cite{uri}. TODO jatka, internet protokollan yhteys http protokollaan?


\begin{figure}[h]
\centering
\includegraphics[scale=0.85]{generic_message.png}
\caption{HTTP-viesti kokonaisuudessaan \cite{httprequest}}
\end{figure}

HTTP-viestit koostuvat asiakaspuolen viesteistä sekä palvelimen vastauksista asiakkaalle. Molemmat viestit sisältävät aloitusrivin, tyhjän tai useamman ylätunnisteen, tyhjän rivin (CRLF) ja mahdollisen lähetettävän viestin. CRLF määrittää rivinvaihdon ja palvelimen on pystyttävä protokollan datavirtaa luettaessa huomioimaan se. Esimerkiksi datavirtaa luettaessa viesti alkaa CRLF:llä se tulee hylätä. HTTP-viesti sisältää erilaisia ylätunnisteita (header), joiden avulla voidaan viedä ylimääräistä tietoa HTTP-pyynnöstä sekä asiakaspäästä. Ylätunnisteita voi olla useita yhdessä HTTP-viestissä. Viestikentässä (message-body) viedään tietyn viestin sisältö, jos sellainen on viestissä mukana \cite{httprequest}. Viestikentän olemassaolo määritetään pituuskentässä (Content-Length) tai kentässä, joka määrittää siirron enkoodauksen (Transfer-Encoding). HTTP-pyynnön ei tule sisältää viestikenttää, jos esimerkiksi enkoodauskenttä löytyy pyynnöstä ja se sisältää minkä tahansa muun arvon kuin "identity".


\chapter{WEB-soketit}



\chapter{Sovelluskehitys}

Katsomalla ympärille nykyajan yhteiskunnassa, huomataan monien asioiden toimivan ohjelmien varassa. Puhelimet, elektroniset kellot, autot ja erilaiset yhteiskunnalliset palvelut sisältävät kaikki ohjelman, jolla toimitoja sekä dataa hallitaan. Nykyajan modernia yhteiskuntaa ei pystytä ylläpitämään ilman sovelluksia. Monet tuotantolinjastot sekä rahoitusjärjestelmät teollisuudessa ovat täysin automatisoituja. Viihdeteollisuus kuten musiikki, pelit, elokuvat ja televisio ovat kaikki monimutkaisten sovellusten varassa. Ihmisten tiedot ovat monissa valtion hallintajärjestelmissä sekä niin julkisissa kuin yksityisissäkin potilastietojärjestelmissä. Nykypäivänä katsottaessa ympärille löydämme lähes poikkeuksetta jonkin asian, jonka taustalla on ohjelma. Suuri määrä ihmisiä kirjoittaa sovelluksia nykypäivänä. Liike-elämässä ihmiset kirjoittavat taulukkolaskentaohjelmia helpottaakseen työtään. Tutkijat ja insinöörit kirjoittavat ohjelmia prosessoimaan kerättyä dataa ja harrastelijat kirjoittavat ohjelmia viihdyttääkseen itseään. Suurin osa sovelluskehityksestä on kuitenkin toteutettu ammattilaisten toimesta tukemaan liiketoimintamalleja, kehittämään laitteita sekä rakentamaan erilaisia palveluita. Ammattilaisten toteuttamat ohjelmistot toteutetaan usein ryhmissä ja ne ovat suunnattu käytettäväksi muille kuin ohjelmoijille itselleen \parencite{Sommerville}.

\section{Sovellusokehityksen määritelmä}
Sovelluskehityksellä tarkoitetaan koko ohjelmiston elinkaaren mittaista tuotantoprosessia alkumäärittelyistä ylläpitoon asti. Se on ammattihaara, joka on erikoistunut tutkimaan ja tuottamaan sovelluksia. Sovelluskehittäjät käyttävät erilaisia teorioita, käytänteitä ja työkaluja ratkaisemaan sovelluskehitykseen liittyviä ongelmia. Sovelluskehitys ei kuitenkaan keskity ainoastaan teknisiin ongelmiin vaan myös projektin hallintaan ja työkaluihin, jotka tukevat sovellusten kehittämistä {Sommerville 1.1.1}. Sovelluskehityksessä tarkoituksena on saada aikaan tuloksia 


\section{Sovellukset}
Ohjelmistot ovat abstrakteja ja aineettomia. Ne eivät noudata mitään fysiikan lakeja eivätkä ne koostu mistään materiasta tai tuotantoprosessista. Tämä tekee ohjelmistokehityksestä helppoa ja vapaata, koska sille ei ole asetettu mitään luonnollisia esteitä. Rajojen puuttuminen johtaa myös siihen, että ne voivat olla usein erittäin monimutkaisia, vaikeita ymmärtää ja kallista muuttaa. Ohjelmistojen tyypit vaihtelevat sulautetuista järjestelmistä maailmanlaajuisiin tietojärjestelmiin. Tietojärjestelmän kehittäminen organisaatiolle eroaa täysin esimerkiksi ohjainlaitteen ohjelmoinnista. Näitä kahta taas ei voi verrata esimerkiksi grafiikkapohjaiseen pelikehitykseen. Kuitenkin kaikki nämä tarvitsevat ohjelmistokehitystä \parencite{Sommenville}. Ohjelmistot eivät ole pelkästään tietokoneohjelmia vaan ne koostuvat ohjelman lisäksi dokumentaatioista sekä erilaisista konfiguraatioista, joilla määritellään ohjelma toimimaan tietyllä tavalla. Dokumentaatio sisältää usein tiedon ohjelmiston rakanteesta sekä käyttäjille suunnatun ohjeen miten sovellusta hallitaan ja käytetään \parencite{Sommerville} section 1.1. Testaus ja sen automaatio on myös suuren osana ohjelmistoja nykypäivänä <lähde tähän pohjaamaan väitettä>. 


Sommenville (1,1) jakaa ammatilaisten kehittämät ohjelmistotuotteet kahteen kategoriaan: geneeriset tuotteet ja räätälöidyt tuotteet. Geneeristen tuotteiden määritelmät ovat sitä kehittävän organisaation hallinnassa. Päätäntävalta ominaisuuksista on siis kehittävällä osapuolella. Räätälöidyissä tuotteissa tilaaja määrittelee ohjelmiston tarkoituksen ja sen ominaisuudet. Näitä kahta tyyppiä voidaan kuitenkin yhdistää, mikä on varsin yleistä ohjelmistoille. Pohjana voi olla geneerinen tuote, johon toteutetaan erilaisia tasoja, jotka ovat määritely tilaajan toimesta \parencite{Sommerville} 

\chapter{Web-sovellukset}
World Wide Web:n kehityksellä on ollut tärkeä rooli sovelluksien kehityksessä. Alunperin Web oli yliopistojen käytössä datan tallennusta varten ja sillä oli hyvin vähän vaikutusta sovelluksiin. Webin ympärille kirjoitetut sovellukset olivat käynnissä paikallisissa tietokoneissa ja niihin oli vain pääsy organisaation sisällä. 2000-luvun alussa Web alkoi kehittyä laajemmin. Selaimiin lisätyt ominaisuudet mahdollistivat Web-pohjaisten sovellusten käytön siten, että erillisen sovellukselle tarkoitetun käyttöliittymän sijasta voitiin käyttää selainta. Tämä johti monien erilaisten sovelluksien kehitykseen, jotka tarjosivat innovatiivisia palveluita käytettäväksi selaimen kautta. Webin kehityksen myötä selaimet kehittyivät siten, että ne pystyivät ajamaan myös pieniä sovelluksia paikallisesti. Tämä muutti organisaatioiden sekä liiketoiminnan tapoja käyttää sovelluksia. Sen sijaan, että sovellukset olisi asennettu käyttäjien tietokoneille, ne voitiin asentaa suoraan yhdelle web-palvelimelle. Tämä laski myös kustannuksia tehdä muutoksia sovelluksiin. Suurin osa on liiketoimien harjottajista on siirtynyt käyttämään web-pohjaisia sovelluksia siellä missä se on mahdollista \cite{Sommerville}.

Seuraava vaihe web-sovellusten kehityksessä olivat web-palvelut. Web-palvelut ovat sovelluskomponentteja, jotka tarjoavat hyödyllisiä palveluja käytettäväksi muille sovelluksille. Siinä missä web-sovellus tarjoaa käyttöliittymän sovellukselle, web-palvelu tarjoaa rajapinnan käyttöliittymäsovellukselle käytettäväksi. Web-palvelu on yksittäinen osa sovellusta ja sovelluksella voi olla monia web-palveluja käytettävänä. 


\section{Ohjelmistojen tyypit}
- lista tyypeist selityksineen
- viite barry boehm "For another thing, unlike the engineering of electrons, materials, or chemicals, the basic software elements we engineer tend to change significantly from one decade to the next. "

\chapter{MVC}
MVC-arkkitehtuurin perusajatus on erottaa käyttöliittymä sovelluslogiikasta ja
näin tehdä sovelluksesta helposti ylläpidettävä kolmen eri komponentin avulla:
Malli (Model), Näkymä (View) ja Ohjain (Controller). Jokainen komponentti on
erikoistunut sovelluksessa johonkin tiettyyn tehtävään. Mallin tehtävänä on
hallita sovelluksen tilaa ja vastata sen käsittelemästä datasta ohjaimelle ja näkymälle.
Näkymän tehtävänä on taas näyttää sovelluksen käyttöliittymä ja sitä kautta mallin dataa. 
Ohjaimen tarkoitus on ottaa vastaan syötteitä käyttäjältä käskien mallia ja näkymää muuttumaan tarvittaessa.

\begin{figure}[h]
\centering
\includegraphics[scale=0.85]{krasner_mvc.jpg}
\caption{Model-View-Controller State and Message Sending \cite[s. 5]{krasner_desc}}
\end{figure} 
Jokaisella komponentilla on oma rajattu tehtävänsä ja ohjelmakoodi tulee jakaa näiden komponentien kesken. Jotta MVC:tä pystyttäisiin käyttämään
tehokkaasti, tulee ymmärtää komponenttien työnjako sekä se kuinka komponentit kommunikoivat keskenään \parencite{burbeck}. 

Luodessamme MVC-arkkitehtuurin toteuttavia komponentteja, tulee ne periä jostakin abstraktista pohjaluokasta (Model, View tai Controller), joka määrittelee kyseisen komponentin käyttäytymisen MVC:ssä  \parencite[s. 5]{krasner_desc}.  Tässä kappaleessa käydään jokaisen komponentin toteutus erikseen läpi käyttäen ohjelmointikielenä Smalltalkia. Lähteenä käytetään Krasnerin julkaisua \parencite{krasner_desc}.

Yleisesti MVC-komponenttien toimintaa kuvaavassa esimerkissä käyttäjältä tulee jokin syöte, jonka sillä hetkellä aktiivinen ohjain ottaa vastaan. Syötteen perusteella ohjain lähettää mallille viestin. Malli puolestaan tekee sille määrättyjä operaatioita muuttaen tilaansa ja lähettää edelleen viestin muutoksestaan kaikille siihen liitetyille riippuvuuksille (näkymät ja ohjaimet). Näkymät
voivat tämän jälkeen kysyä mallilta sen nykyistä tilaa ja päivittää itsensä, jos siihen on tarvetta. Ohjaimet voivat myös muuttaa tilaansa riippuen mallin tilasta \parencite[s. 4]{krasner_desc}. 

Suurin merkitys MVC:llä on luoda silta ihmismielen hahmottamalle mallille ja tietokoneessa esiintyvälle mallille. Oikein toteutettuna MVC:n avulla luodaan illuusio siitä, että käyttäjä kommunikoi suoraan mallin kanssa. Todellisuudessa kuitenkin ohjain ja näkymä muodostavat yhdessä rajapinnan sille, miltä malli näyttää ulospäin ja miten sitä käsitellään. Ohjain huolehtii syötteiden vastaanottamisesta ja käsittelemisestä. Näkymä taas huolehtii mallin graafisesta puolesta \parencite[s. 11-12]{reenskaug_tools}. 

\section{Historia}
MVC:n esitteli Norjalainen Trygve Reenskaug ollessaan mukana Xerox PARC -tutkimushankkeessa. 
Ensimmäinen julkaisu MVC:stä kirjoitettiin vuonna 1978 samassa tutkimuskeskuksessa. 
Tuolloin julkaisussa esiteltiin kolmen komponentin sijasta neljä komponenttia: 
Malli (Model), Näkymä(View), Ohjain(Controller) sekä Muokkaaja(Editor). Muokkaaja on 
väliaikainen komponentti, jonka näkymä luo itsensä ja syötelaitteiden välille. 
Muokkaaja-komponentista kuitenkin luovuttiin käsitteenä ja se sisällytettiin näkymään 
ja ohjaimeen \parencite{xerox}. Alkuperäinen Xerox PARC:n tuottama raportti MVC:stä oli Reenskaugin 
vuonna 1979 kirjoittama THING-MODEL-VIEW-EDITOR \parencite{xerox-thing}. Raportti esitteli MVC:n 
komponentteja käyttäen hyväksi esimerkkejä Reenskaugin omasta suunnittelutyöstä. Thing-komponentilla mallinnettiin
jotakin isompaa kokonaisuutta, joka hallitsee pienempiä kokonaisuuksia. Sitä voidaan ajatella eräänlaisena suurena mallina,joka on jaettu useisiin pienempiin malleihin. Editor-komponentti luo rajapinnan käyttäjän ja yhden tai useamman näkymän välille. Se tarjoaa käyttäjälle sopivan komento-rajapinnan kuten esimerkisi valikon, joka vaihtuu sisällön muuttuessa \parencite{xerox-thing}. Reenskaug hylkäsi kuitenkin Editor- ja Thing-komponentin ja  päätyi Adele Goldbergin avustuksella termeihin Models-Views-Controllers julkaisten saman vuoden lopulla raportin, jossa määritellään lyhyesti jokaisen komponentin tehtävä (MODELS-VIEWS-CONTROLLERS) \parencite{xerox-original}. Koska MVC:n historia ja suurin osa MVC:n alkuperäisistä julkaisuista pohjautuvat Smalltalk-ohjelmointikieleen, esitellään myös tässä tutkielmassa MVC:n totetusta Smalltalkilla.
Tämä ei kuitenkaan rajoita tarkastelua, koska arkkitehtuurin idea pysyy täysin samana riippumatta ohjelmointikielestä.

\section{Malli (Model)}
Malli pitää yllä sovelluksen tilaa sekä vastaa sovelluksen tallentamasta datasta. Se voi olla esimerkiksi kokonaislukumuuttuja laskuri-sovelluksessa, merkkijono-olio tekstinkäsittelyohjelmassa tai
mikä tahansa monimutkainen olio \parencite[s. 3]{krasner_desc}. Kaikkein yksinkertaisimmassa tapauksessa mallin ei tarvitse kommunikoida ollenkaan ohjaimen ja näkymän kanssa, vaan toimia passiivisena säiliönä datalle.
Tällaisesta tilanteesta on hyvä esimerkki yksinkertainen tekstieditori, jossa teksti nähdään juuri sellaisena kuin se olisi paperilla. Tässä tapauksessa mallin ei tarvitse ottaa vastuuta
kommunikoinnista näkymälle, koska muutokset tekstiin tapahtuvat käyttäjän pyynnöstä. Tällöin ohjain ottaa vastaan käyttäjän syötteet ja voi esimerkiksi ilmoittaa näkymälle muutoksesta, jolloin näkymä
päivittää mallin. Ohjain voi myös päivittää mallin ja ilmoittaa tästä näkymälle, jolloin näkymä voi pyytää mallin sen hetkistä tilaa. Kummassakaan tapauksesssa mallin ei tarvitse tietää ohjaimen ja näkymän
olemassaolosta \parencite{burbeck}.

Malli ei kuitenkaan aina voi olla täysin passiivinen. Se voi myös muuttua ilman, että se tarvitsee ohjaimen tai näkymän käskyä. Otetaan esimerkiksi malli, joka muuttaa tilaansa satunnaisin väliajoina. Koska malli muuttaa itseään, täytyy sillä olla jokin yhteys näkymään, jotta se voi antaa tiedon muutoksestaan \parencite{burbeck}. Datan kapseloinnin ja ohjelmakoodin uudelleen käytön kannalta ei ole kuitenkaan järkevää, että malli on suoraan yhteydessä näkymään ja ohjaimeen. Ohjaimen ja näkymän tulee siis olla riippuvaisia mallista, mutta ei toisinpäin. Näin mahdollistetaan myös se, että mallilla voi olla useita näkymiä ja ohjaimia \parencite[s. 4]{krasner_desc}.

Yleensä mallin tila muuttuu ohjaimista tulleiden käskyjen kautta. Tämän muutoksen tulisi heijastua kaikkiin näkymiin, jotka ovat sidottuja malliin. Tällaisia tilanteita varten kehitettiin riippuvuudet (\emph{dependents}).
Riippuvuuksilla tarkoitetaan listaa niistä ohjaimista ja näkymistä, jotka ovat sidottuja malliin. Mallilla tulee siis olla lista riippuvuuksista ohjaimiin ja näkymiin sekä myös kyky lisätä ja poistaa niitä. Malli ei siis tiedä mitään yksittäisistä riippuvuuksista, mutta pystyy kuitenkin lähettämään itsestään muutosviestejä (\emph{change messages}) listassa oleville ohjaimille ja näkymille. Mallin tuottamat muutosviestit voivat olla minkä tyyppisiä tahansa, joten ohjaimet ja näkymät reagoivat niihin omalla määritellyllä tavallaan  \parencite[s.2-3]{krasner}.

Mallille määritellään pääluokka \emph{Model} ja tälle viitemuuttuja \emph{dependents}, joka viittaa yhteen riippuvaan komponenttiin tai listaan riippuvista komponenteista. Kaikki uudet mallit tulee periä niiden pääluokasta, jotta saavutetaan sama toiminnallisuus kaikkiin mallikomponentteihin. Komponenttien tieto mallin muutoksista tukeutuu täysin mallin riippuvuusmekanismiin. Kun jokin komponentti luodaan, se rekisteröi itsensä malliin riippuvuudeksi ja samalla tavalla se myös poistaa itsensä \parencite{burbeck}. Näkymät käyttävät riippuuvuusmekanismia päivittääkseen itsensä mallin muutoksien perusteella. Esimerkiksi mallin muuttuessa lähetätetään \textit{changed}, jonka pohjalta jokainen riippuvuus saa \textit{update} -viestin.Viestillä voi olla myös erilaisia parametrejä, joiden perusteella viestiä pystytään tarkentamaan. Esimerkiksi mallin, johon on liitetty useita näkymiä, ei välttämättä tarvitse lähettää kaikille näkymille viestiä muutoksestaan. Se voi välittää viestin mukana parametrina tiedon muutoksesta, jonka perusteella jokainen vastaanottaja voi päättää miten toimia \parencite{burbeck}.

Alkuperäinen \textit{update} -metodi on peritty \textit{Object} -luokasta, eikä se tuolloin tee vielä yhtään mitään. Useimmilla näkymillä se on kuitenkin toteutettu näyttämään näkymä uudestaan kutsuttaessa. Tämä \textit{changed/update} -mekanismi valittin toimimaan kommunikaatiokanavana mallien ja näkymien välille, koska se aiheuttaa vähiten rajoituksia ja esteitä \parencite{burbeck}. 

\subsection{Näkymä (View)}
Näkymän tehtävänä on huolehtia graafisesta puolesta MVC:ssä. Näkymä pyytää yleensä mallilta datan ja tämän pohjalta näyttää käyttäjälle käyttöliittymän sovellukseen. Toisinkuin malli, jota pystytään rajoittamattomasti yhdistelemään moniin näkymiin ja ohjaimiin, jokainen näkymä on liitetty yhteen ohjaimeen.  Näkymä siis sisältää viitteen ohjaimeen ja ohjain sisältää viitteen näkymään. Kuten ohjain, näkymä on myös rekisteröity mallin riippuvuuksiin. Kummatkin sisältävät siis myös viitteen siihen malliin, johon ne on rekisteröity \parencite{burbeck}. Jokaisella näkymällä on tasan yksi malli ja yksi ohjain \parencite[s. 7]{krasner_desc}.

Näkymä vastaa myös MVC-komponenttien sisäisestä kommunikaatiosta MVC-kolmikon luontivaiheessa. Näkymä rekisteröi itsensä  riippuvuudeksi malliin, asettaa viitemuuttujansa viittamaan ohjaimeen ja välittää itsestään viestin ohjaimelle. Viestin avulla ohjain rekisteröi näkymän omaan viitemuuttujaansa. Näkymällä on myös vastuu poistaa viitteet sekä rekisteröinnit \parencite{burbeck}. 

Näkymä ei sisällä ainoastaan komponentteja datan näyttämiseen ruudulla, vaan se voi sisältää myös useita alanäkymiä (\emph{subviews}) ja ylänäkymiä (\emph{superviews}). Tästä muodostuu hierarkia, jossa ylänäkymä hoitaa aina jonkun suuremman kokonaisuuden, kuten esimerkiksi näytön pääikkunan. Alanäkymä taas huolehtii jostain pienemmästä yksityiskohdasta pääikkunassa. Näkymillä on myös viite erilliseen transformaatioluokkaan, joka hoitaa kuvan sovittamisen ja yhdistämisen alanäkymien ja ylänäkymien välillä. Jokaisella näkymällä tulee siis olla toteutus, jolla hoidetaan alanäkymien poistaminen sekä lisääminen. Samalla tulee määritellä ominaisuus, jolla sisäiset transformaatiot tuodaan transformaatioluokalle. Tämä helpottaa näkymän ja sen alanäkymien yhdistämistä \parencite[s. 8]{krasner_desc}. burbeck havainnollistaa Smalltalkilla kirjoitetulla esimerkillä kuinka MVC-kolmikko luodaan. Esitetyssä esimerkissä on yksinkertaistettu versio MVC-kolmikon luonnista siten, että mukana on myös ylä- ja alanäkymien toteutus.

\begin{lstlisting}[language=Smalltalk]
openListBrowserOn: aCollection label: labelString initialSelection: sel
  "Create and schedule a Method List browser for 
  the methods in aCollection."
  | topView aBrowser | 
  aBrowser := MethodListBrowser new on: aCollection.
  topView := BrowserView new.
  topView model: aBrowser; controller: StandardSystemController new;
                 label: labelString asString; minimumSize: 300@100.
topView addSubView:
  (SelectionInListView on: aBrowser printItems: false oneItem: false
  aspect: #methodName change: #methodName: list: #methodList
  menu: #methodMenu initialSelection: #methodName)
  in: (0@0 extent: 1.0@0.25) borderWidth: 1.
topView addSubView:
  (CodeView on: aBrowser aspect: #text change: #acceptText:from:
  menu: #textMenu initialSelection: sel)
  in: (0@0.25 extent: 1@0.75) borderWidth: 1.
  topView controller open
\end{lstlisting}

Seuraavaksi käydään rivi kerrallaan läpi mitä yllä esitetyssä ohjelmakoodissa tapahtuu. Mallin luonnin jälkeen [5] luodaan viite uudelle \textit{BrowserView} -luokan instanssille [6]. \textit{BrowserView} on peritty \textit{StandardSystemView} -luokasta. Seuraavaksi määritellään malli ja ohjain sekä muuttujat näkymän otsikolle ja koolle [7]. Jos ohjainta ei määritellä erikseen, käytetään näkymän \textit{defaultController} metodia. Riveillä [7-11] luodaan alanäkymä \textit{SelectionInListView} ja riveillä [12-15] luodaan toinen alanäkymä \textit{CodeView}. Lopuksi [16] avataan ohjain, joka käynnistää ikkunoiden piirtämisprosessin.

Näkymät saattavat tarvita myös oman protokollan itsensä näyttämiseen. Kun malli ilmoittaa muutoksestaan, \textit{update} -metodi näkymässä kutsuu \textit{display}, joka puolestaan kutsuu \textit{displayBorder}, \textit{displayView} ja \textit{displaySubviews}. Jos näkymä tarvitsee erityistä käyttäytymistä itsensä näyttämiseen, se toteutetaan edellämainituissa metodeissa. Muuten käytetään pääluokasta perittyjä ominaisuuksia \parencite{burbeck}. Monet näkymät käyttävät myös erilaisia transformaatio-instansseja, joilla hallitaan esimerkiksi näkymän skaalausta ruudulla. Tähän ei kuitenkaan perehdytä sen enempää, koska ne menevät tutkimuksen rajojen ulkopuolelle.


\subsection{Ohjain (Controller)}
Ohjaimen tehtävänä on ottaa vastaan syötteitä sekä koordinoida malleja ja näkymiä saatujen syötteiden perusteella. Sen tulee myös kommunikoida muiden ohjaimien kanssa. Teknisesti ohjaimessa on kolme viitemuuttujaa: malli, näkymä ja sensori (sensor). Sensorin tehtävänä on toimia rajapintana syötelaitteiden sekä ohjaimen välillä. Sensori mallintaa syötelaitteiden käyttäytymistä ja muuttaa ne ohjaimen ymmärtämään muotoon.

Ohjaimien tulee käyttäytyä siten, että vain yksi ohjain ottaa vastaan syötteitä kerrallaan. Esimerkiksi näkymät pystyvät esittämään informaatiota rinnakkain monen näkymän kautta, mutta käyttäjän toimintoja tulkitsee aina vain yksi ohjain. Ohjain on siis määritelty käyttäytymään siten, että se osaa tietyn signaalin perusteella päättää tuleeko sen aktivoida itsensä vai ei. Ohjain sisältää toiminnallisuuden jonka perusteella se pystyy päättämään tuleeko hallinta pitää itsellä vai luovuttaa eteenpäin \parencite[s. 9]{krasner_desc}. Ohjainten ylimmällä tasolla on \textit{ControlManager}, joka kysyy jokaiselta päänäkymään liitetyltä ohjaimelta erikseen, haluaako tämä ottaa hallinnan. Jos ohjaimen näkymä sisältää kursorin, vastaa ohjain kutsuun myönteisesti, jolloin kyseinen ohjain saa hallinnan. Hallitsevan ohjaimen näkymä kysyy seuraavaksi mahdollisten alanäkymien ohjaimilta samalla tavalla haluaako jokin ohjaimista hallinnan itselleen. Jos myönteisesti vastaava ohjain löytyy, ottaa se uuden hallinnan. Tätä prosessia jatkamalla löydetään matalimman tason näkymä ja sen ohjain ottaa lopullisen hallinnan. Ohjain pitää hallinnan itsellään niin kauan kunnes kursoria liikutetaan näkymän rajoista ulos. Ainoastaan se jonka kohdalla kursori on, vastaa kutsuun ja tuolloin ottaa hallinan. Näkymillä on oikeus kysyä alanäkymiensä ohjaimia. Ohjaimien tehtävänä on kysyä omalta näkymältään onko kursori niiden päällä.

Krasner määrittelee seuraavat metodit, joiden avulla ohjaimet viestivät \parencite[s. 9]{krasner_desc}:

\begin{description}
\item[isControlWanted] -\ Tuleeko ohjaimen ottaa hallinta.
\item[isControlActive] -\ Onko ohjain aktiivinen.
\item[controlToNextLevel] -\ Luovutetaan hallinta seuraavalle ohjaimelle.
\item[viewHasCursor] -\ Onko ohjaimen näkymässä hiiren kursori.
\item[controlInitialize] -\ Kun ohjain on saanut hallinnan, alustetaan se.
\item[controlLoop] -\ Lähettää \emph{controlActivity} -viesteja niin kauan, kuin ohjaimella on hallinta.
\item[controlTerminate] -\ Lopettaa ohjaimen hallinnan.
\end{description} 

Kun ohjain saa hallinnan itselleen, kutsuu se \emph{startUp} -metodia, joka puolestaan kutsuu seuraavia metodeja: \emph{controlInitialize}, \emph{controlLoop} ja \emph{controlTerminate}. Metodit
voidaan ylikirjoittaa, jolloin saavutetaan jokin haluttu ominaisuus kyseisessä vaiheessa. Esimerkiksi \emph{controlInitialize} ja \emph{controlTerminate} määräävät mitä tehdään, kun ohjain saa hallinnan tai luovuttaa sen eteenpäin. Ohjaimen hallinnan aikana kutsutaan
\emph{controlLoop} -metodia, joka taas kutsuu \emph{controlActivity} -metodia niin kauan kuin ohjaimella on hallinta.  Metodi \emph{controlActivity} määrää ohjaimen toiminnan hallinnan aikana \cite[s. 9]{krasner_desc}.

\subsection{Esimerkkiohjelma}
Seuraavaksi esitellään  Dortmundin yliopistossa kirjoitettu yksinkertainen esimerkkiohjelma Smalltalkilla siitä miten MVC:n toteutus tuodaan sovellukseen käytännössä. Ohjelmakoodi löytyy myös Krasnerin artikkelista \cite[s. 20]{krasner_desc}. Ohjelmassa toteutetaan yksinkertainen laskuri-ohjelma, joka käyttää MVC-arkkitehtuuria toteutuksessaan. Ohjelmassa esitellään mallina \emph{Counter} -luokka ja näkymänä \emph{CounterView} -luokka. \emph{Counter} perii mallin ominaisuudet ja
toimii ohjelmassa yksinkertaisen kokonaisluku-muuttujan ylläpitäjänä. \emph{CounterView} perii näkymän ominaisuudet ja esittää mallin arvon ruudulla.
Ohjaimena toimii \emph{CounterController} -luokka, joka perii ohjaimen käyttäytymisen. Ohjain tarjoaa sovellukselle painikkeet, joista voidaan vähentää
tai lisätä laskurin arvoa.

Määritellään ensiksi \emph{Counter} -luokka, joka peritään \emph{Model} -luokasta.
\begin{lstlisting}[language=Smalltalk]
Model subclass: #Counter
	instanceVariableNames: 'value'
	classVariableNames: ''
	poolDictionaries: ''
  	category: 'Demo-Counter'
\end{lstlisting}
Seuraavaksi määritellään \emph{Counter}-luokalle metodeita, jotka määrävät
laskuriarvon alustamisen sekä muokkaamisen.
\begin{lstlisting}[language=Smalltalk]
Counter methods For: 'Initialize-release'
Initialize
	"Aseta alkuarvoksi 0"
	self value: 0
Counter methodsFor: 'accessing'
value
	"Palauta mallin arvo"
   	^value
value: aNumber
	"Aseta mallin arvo"
	value <- aNumber.
	self changed "to update displayed value"
Counter methodsFor: 'operations'
decrement
	"Vähennä mallin arvoa yhdellä."
	self value: value -1
Increment
	"Lisää mallin arvoa yhdellä."
	self value: value + 1
\end{lstlisting}
Lisätään luokkaan metodi, jolla itse luokasta saadaan muodostettua instanssi.
\begin{lstlisting}[language=Smalltalk]
Counter class methodsFor: 'instance creation'
new
	"Palauta uusi instanssi luokasta"
	^super new initialize
\end{lstlisting}
Seuraavaksi määritellään ohjain (\emph{CounterController}), joka peritään
\emph{Controller} luokasta. Luodaan myös ohjaimelle metodit, joiden avulla
ohjataan mallia sekä näkymää. Metodeissa toteutetaan valikko, joka tarjoaa
mahdollisuuden joko vähentää tai lisätä laskurin arvoa. Kaikki \emph{CounterController} -luokassa käytetyt
määrittelemättömät muuttujat peritään yliluokasta.
\begin{lstlisting}[language=Smalltalk]
Mouse MenuController subclass: #CounterControIler
	instanceVariableNames: ' '
  	classVariableNames: ' '
  	poolDictionaries: ' '
  	category: 'Demo-Counter'
CounterController methodsFor: 'initialize-release'
initialize
	"Alusta valikko, jossa on mahdollisuus vähentää tai 
        lisätä mallin arvoa"
  	super initialize.
  	Self yellowButtonMenu: (PopUpMenu labels: 
                                  'Increment\Decrement' withCRs)
  	yellowButtonMessages: #(increment decrement)
CounterController methodsFor: 'menu messages'
decrement
	"Vähennä mallin arvoa yhdellä."
 	self model decrement
increment
	"Lisää mallin arvoa yhdellä"
	self model increment
CounterController methodsFor: 'control defaults'
isControlActlve
	"Ota hallinta kun sinistä nappia ei paineta"
	^super isControlActive & sensor blueButtonPressed not
\end{lstlisting}
Määrätään näkymä (\emph{CounterView}), joka peritään \emph{View} -yliluokasta. Määrätään
myös näkymälle metodit, joiden avulla näytetään mallin tila ruudulla.
\begin{lstlisting}[language=Smalltalk]
View subclass: #Counterview
	instanceVariableNames: ''
	classVariableNames: ''
	poolDictionaries: ''
	category: 'Demo-Counter'

CounterView methodsFor: 'displaying'
displayView
	"Näytä mallin arvo näkymässä"
	| box pos displayText |
	box := self insetDisplayBox.
	"Asettele teksti näkymään. Asettelu ei
	 ole tutkielman kannalta oleellista."
	pos := box origin + (4 @ (box extent y / 3)).
	displayText := ('value:', self model value printString)
					asDisplayText.
	displayText displayAt: pos
\end{lstlisting}
Määritellään \emph{update} -metodi, jotta näkymä pystyy päivittämään itsensä. Metodia kutsutaan
yleensä mallin tilan muuttuessa.
\begin{lstlisting}[language=Smalltalk]
CounterView methodsFor: 'updating'
update: aParameter
  "Yksinkertaisesti päivitä näyttö uudestaan"
  self display
\end{lstlisting}
Luodaan myös metodi, joka palauttaa näkymään liitetyn ohjaimen.
\begin{lstlisting}[language=Smalltalk]
CounterView methodsFor: 'controller access'
defaultControllerClass
	"Palauta näkymään rekisteröity ohjain"
	^CounterController
\end{lstlisting}
Lopuksi tarvitaan metodi, joka luo uuden näkymän sekä rekisteröi mallin ja ohjaimen itseensä. Näkymä näyttää ruudulta
samalta kuin kuvassa 2.
\begin{lstlisting}[language=Smalltalk]
CounterView class methodsFor: 'instance creation'
open
	"Avaa näkymän uudelle laskurisovellukselle. Tässä 
	metodissa nähdään kuinka näkymä huolehtii mallin 
	rekisteröinnistä sekä nähdään kuinka näkymiä voi 
	olla useita sisäkkäin."
	| aCounterView topView |
	"Luo laskurinäkymälle uusi näkymä, joka näyttää 
	laskurin arvon"
	aCounterView := CounterView new
	"Asetetaan malliksi Counter -luokan instanssi"
	model: Counter new.
	aCounterView borderWidth: 2.
	aCounterView insideColor: Form white.
	"Asetetaan ylimmäksi näkymäksi StandardSystemView 
	-luokan instanssi, joka vastaa perinteistä
ikkunointimallia"
	topView := StandardSystemView new
		label: 'Counter'.
	topView minimumSize: 80@40.
	"Lisätään edellä luotu laskurinäkymä ylinäkymän 
	alanäkymäksi"
	topView addSubView: aCounterView.
	"Käynnistetään ohjain"
	topView controller open
\end{lstlisting}

\begin{figure}[h]
\centering
\includegraphics[scale=0.85]{counter.png}
\caption{Kuva CounterView -näkymästä \cite{krasner_desc}}
\end{figure}

\chapter{Sovelluskehykset}
Soveluskehykset ovat suosittuja, koska ne tarjoavat uudelleenkäytettäviä ratkaisuja erilaisiin ongelmiin sovelluskehityksessä. Toimialueesta riippumatta sovelluskehyksiä tulisi käyttää hyväksi kirjottaessa monimutkaisia sovelluksia. Sovelluskehys tuo sovellukseen tason, jossa sovelluksen osat on abstrahoitu erilaisilla luokilla sekä rajapinnoilla, joita voidaan käyttää uudelleen sovelluksen eri osissa. Sovelluskehys ei ole vain kokoelma rajapintoja ja kirjastoja\parencite{towards_framework}. Tärkein ero sovelluskehyksen ja kirjaston välillä on se, että kirjaston ohjelmakoodi kutsustaan aina
kehittäjän toimestaa. Sovelluskehyksessä taas kehittäjän ohjelmakoodia kutsutaan aina sovelluskehyksen toimesta \parencite{pyramid_intr}.

Sovelluskehys ei myöskään generoi koodia. Se käyttää erilaisia komponentteja ja kirjastoja luodakseen infrastruktuurin, jonka päälle voidaan rakentaa sovelluksia sovelluskehyksen ehdoilla. Sovelluskehyksen käyttäminen myös rajoittaa sovelluksen rakennetta ja pakottaa sovelluksen toteuttamaan asioita tietyin ehdoin. Rajoitusten ansiosta sovelluskehittäjä voi keskittyä toimialueeseen liittyviin ongelmiin välittämättä koko sovelluksen yksityiskohtaisesta toteutuksesta \parencite{towards_framework}.

Web-sovelluskehykset ovat sovelluskehyksiä, jotka tarjoavat ratkaisuja helpottamaan web-sovellusten toteuttamista. Web-sovelluskehyksissä käyttöliittymä näytetään käyttäjille selaimen välityksellä. Sovellus ajetaan joko serverillä tai suoraan käyttäjän selaimessa. Sovellus määrittää käyttöliittymän sivujen järjestyksen, sisällön sekä mahdollisten toimintojen esittämisen käyttäjälle, jonka kautta käyttäjä voi vaikuttaa serverillä sijaitsevaan sovellukseen \parencite{vosloo}. 

Yleisimmät teknologiat mitä web-sovelluskehykset tarjoavat ovat rajapinta tietokannalle, template-moottori sekä mahdollisuus käsitellä http-pyyntöjä ohjelmakoodissa. Tietokanta-rajapinnalla tuodaan sovelluskehitykseen taso, jonka avulla helpotetaan kommunikointia tietokannan kanssa. Yleisimmin käytetty ohjelmointitekniikka tähän on ORM (Object-Relational-Mapping), jolla muunnetaan dataa tietokannan ja ohjelmakoodin välillä. Ohjelmoijalle tämä näkyy ns. virtuaalisena olio-tietokantana, jonka avulla voidaan lukea sekä muokata tietokantaa kutsuilla ohjelmakoodista. Tällöin suoria kyselyitä tietokantaan ei tarvita \parencite{Ghandeharizadeh}. Seuraavassa esimerkissä esitellään miten Djangon ORM:ia käytetään.

\begin{lstlisting}[language=Python]
from django.db import models

class Blog(models.Model):
    name = models.CharField(max_length=100)
	tagline = models.TextField()
\end{lstlisting}

\begin{lstlisting}[language=Python]
from blog.models import Blog

b = Blog(name='Beatles Blog', tagline='All the latest Beatles news.')
b.save()
\end{lstlisting}

Esimerkissä luodaan luokka, jolla määritellään taulu SQL-tietokantaan. Model-luokasta periminen mahdollistaa luokasta luotujen instanssien tallentumisen tauluun. Lisäksi
luokalle voidaan määrittää attribuutteja, jotka käyttävät Djangon tarjoamia kenttiä. Kentät vastaavat tietokannassa olevia data-tyyppejä. Lopullinen tallentuminen
tapahtuu kutsumalla instanssin save() -metodia.


\begin{lstlisting}[language=Python]
beatles_blog = Blog.objects.get(name="Beatles Blog")
\end{lstlisting}

Yllä oleva rivi hoitaa kyselyn nimen perusteella blog -nimisestä taulusta, luo sen kenttien perusteella instanssin Blog -luokasta ja tallentaa sen muuttujan arvoksi.


Template-moottori on teknologia HTML-sivujen tuottamiseen, jolla generoidaan dynaamisia HTML-sivuja yhdistämällä ohjelmalogiikkaa sekä HTML-kieltä. Alla on esitelty Jinja2-kieli, jota käytetään template-kielenä Flask:ssa.
\pagebreak

\begin{lstlisting}[language=Smalltalk]
<title></title>
<ul>
  
	<li><a href="{{ user.url }}">{{ user.username }}</a></li>
  
</ul>
\end{lstlisting}

Yllä esitellyssä esimerkissä luodaan HTML-sivu, jossa tulostetaan otsikko sekä lista url-osoitteita sekä käyttäjänimiä jokaista käyttäjää kohden \parencite{jinja}.

Web-sovelluskehykset voidaan jakaa serveri-pohjaisiin ja selain-pohjaisiin sovelluskehyksiin. Serveri-pohjaisissa web-sovelluskehyksissä sovelluksen tilaa hallitaan serverin puolella. Tuolloin käyttöliittymää perustuu tilaan, mikä on sillä hetkellä serverin puolella. Selain-pohjaisissa sovelluskehyksissä sisältö muuttuu selaimen sisällä käyttäjän päässä \parencite{vosloo}. Tässä tutkimuksessa käsiteltävät sovelluskehykset ovat Pythonilla toteutettuja serveri-pohjaisia sovelluskehyksiä. Käsiteltävät kehykset ovat Django, Pyramid ja Flask. Django on käsiteltävistä sovelluskehyksistä kaikkein monoliittisin ja tarjoaa eniten ominaisuuksia valmiiksi asennettuna. Pyramidissa taas on vähemmän ominaisuuksia suoraan asennettuna, jonka kautta se pyrkii antamaan käyttäjälle enemmän valinnanvaraa erilaisten teknologioiden valitsemiseen. Flaski taas on mikro-sovelluskehys, joka tarjoaa kaikkein vähiten työkaluja web-kehitykseen näistä kolmesta. Sovelluskehysten koko kertoo myös niiden toteutuksesta: Django (1.7.4) 36 megabittiä, Pyramid (1.5.3) 5.6 megabittiä ja Flask (0.10.1) 1.2 megatavua. Sovelluskehysten koot tarkastettiin komentoriviltä käyttäen linuxin du -työkalua. Kaikkia kolmea sovelluskehystä voidaa laajentaa erilaisilla laajennoksilla.


\section{Pyramid}

Pyramid on Python-pohjainen web-sovelluskehys, jonka tehtävänä on helpottaa web-kehitystä tarjoamalla
kehittäjälle valmiita työkaluja avuksi kehitykseen. Pyramid on suunniteltu siten, että kehittäjän ei tarvitse tietää suuria määriä erilaisia malleja ja tekniikoita pystyäkseen tuottamaan web-sovelluksia. Se ei myöskään pakota käyttämään kehityksessä mitään erityistä tekniikkaa, vaan pyrkii olemaan mahdollisimman yksinkertainen ja helposti laajennettavissa erilaisiin käyttötarkoituksiin. Laajentamisella tarkoitetaan erilaisten lisäosien liittämistä Pyramidiin. Yksinkertaisuuden ja mimimaalisuuden ansiosta se on myös nopeampi kuin monet muut Python-pohjaiset web-sovelluskehykset. Tämä johtuu Pyramidin poikkeuksellisen pienestä kutsupinosta ajamisen aikana \parencite{pyramid_intr}. 

Pyramid sai alkunsa Pylons -projektista syyskuussa vuonna 2005, jolloin jo yli 30 Python-sovelluskehystä kilpaili käyttäjistä. Ben Bangert ja James Gardner alkoivat yhdessä kehittää sovelluskehystä, josta tuli lopulta Pylons. Alunperin Pylons oli
muokattu Myghty Python Templating Framework:n pohjalta tarjotakseen MVC-pohjaisen web-sovelluskehyksen. Myghty -sovelluskehystä ajettiin mod\_pythonin päällä, mutta Pylonsin pyrkimys oli käyttää WSGI:tä hyödyntämään joustavaa komponenttipohjaista lähestymistapaa web-sovelluksissa \parencite{pyramid_history}. Pylons projektin tarkoituksena ei ole keskittyä yhden yksittäisen web-sovelluskehyksen kehittämiseen vaan tarjota kokoelma erilaisia teknologioita \parencite{pylons_about}. Vuonna 2008 Pyramid tunnettiin nimellä repoze.bfg. Joulukuun alussa tapahtui ohjelmakoodin uudelleen nimeäminen ja ominaisuuksien lisääminen sekä poistaminen \parencite{pyramid_about}.

Koska Pyramid pyrkii tarjoamaan vain välttämättömimmät työkalut web"-sovelluksien kehitykseen, sen kehittäjät
ovat päätyneet web-kehityksessä neljään yleisimpään ongelmaan ja tarjoavat niihin ratkaisun Pyramidissa:

\begin{description}
\item [URL Mapping] -\ URL:ien liittäminen ohjelmakoodiin.
\item[Template] -\ Tuodaan sovelluksen näkymä selaimelle käyttäen template-kieltä, jolla määrätään näkymän rakenne. Kieli on usein HTML:än mukana tuotuja loogisia ilmaisuja, joiden perusteella template-engine rakentaa HTML-sivun selaimelle. Templatejen avulla pystytään erottamaan käyttöliittymä sovelluslogiikasta tehokkaasti.
\item[Security] -\ Perinteiset tietoturvaongelmat tulee olla ratkaistuna valmiiksi jo sovelluskehyksessä. Tämä ei kuitenkaan tarkoita sitä, että kehittäjä voisi täysin unohtaa tietoturvan merkityksen.
\item[Static Assets] -\ Staattisien resurssejen jakaminen niille tarkoitettuihin paikkoihin tiedostorakenteessa.
\end{description}
Yllä määriteltyjen neljän ongelman lisäksi Pyramid tarjoaa lisäosien kautta monia erilaisia työkaluja, joiden avulla pystytään laajentamaan sen ominaisuuksia \parencite{pyramid_intr}. Tutkielman aiheen rajauksen vuoksi ei kuitenkaan
käydä läpi yksityiskohtaisemmin Pyramidin toteutusta ja siihen liitettävissä olevia lisäosia, vaan keskitytään tarkastelemaan MVC:n toteutusta Pyramidissa.

\section{Django}
Django on web-sovelluskehys, joka  sai alkunsa kehitysryhmässä Kansaksen osavaltiossa Yhdysvalloissa 2003, kun web-kehittäjät Adrian Holovaty ja Simon Willison alkoivat käyttää Pythonia web-kehityksessä. The World Online -ryhmä (WO), joka oli vastuussa muutamaamasta paikallisesta uutissivustosta, menestyivät ympäristössä, jossa oli tiukat aikarajat. Journalistit vaativat ominaisuuksien ja kokonaisten sovelluksien valmistumista muutamassa päivässä ja joskus jopa tunneissa. Holovaty ja Willison kehittivät web-sovelluskehyksen, jonka avulla he pystyivät vastaamaan journalistisen ympäristön haasteisiin. Kesällä 2005 he saivat kehitettyä sovelluskehyksen siten, että se oli käytössä suurimmassa osassa World Onlinen sivustoja. Tuolloin mukaan kehitykseen tuli Jacob Kaplan-Moss. Kehittäjät päättivät julkaista heinäkuussa 2005 sovelluskehyksen nimellä Django jazz-kitaristi Django Reinhardtin mukaan \parencite{django_history}.

Django tarjoaa samat välttämättömät työkalut kuin Pyramidissa. Se tarjoaa myös ylläpitäjille suunnatun työkalun, josta voidaan hallita sovellusta käyttöliittymätasolta. Lisäksi se tarjoaa lomake-työkalut, käyttäjätasoisen autentikaation sekä tietokanta-abstrahoinnin. Tietokanta-abstraktiolla tarkoitetaan sisäänrakennettua virtuaalista ympäristöä tuomaan yhteys tietokantaan olio-ohjelmoinnin kautta (Object-relational mapping)\parencite{djangobook}. Siinä missä Pyramid pyrkii tarjoamaan kehittäjille valinnanvaraa erilaisten komponenttejen suhteen, Django tarjoaa kokonaisvaltaisen ratkaisun sisältäen kaikki tarvittavat työkalut web-sovellusten rakentamiseen. Djangoon on tarjolla myös paljon erilaisia paketteja täydentämään sitä. 

\section{Flask}
Flask on mikro-sovelluskehys (micro-framework), jossa sovelluskehyksen lähdekoodi on pidetty mahdollisimman minimalistisena. Sen tarkoituksena ei ole tehdä valintoja kehittäjän puolesta millaisia teknologioita tämän tulisi käyttää. Ne teknologiat mitä Flask tuo mukanaan, voidaan korvata toisella vastaavalla teknologialla. Esimerkiksi flask tuo mukanaan Jinja2 template-kielen, joka voidaan korvata millä tahansa muulla template-kielellä.

Flaski ei tuo mitään lomake-työkaluja, tietokanta-abstrahointeja tai mitään missä ulkoinen kirjasto ei pystyisi kyseistä teknologiaa toteuttamaan. Flask tukee laajennoksia, joilla sitä voidaan laajentaa erilaisilla teknologioilla. Tällaisia ovat esimerkiksi erilaiset lomake-kirjastot ja tietokanta-abstrahoinnit. Flask on tämän tutkimuksen sovelluskehyksistä minimaalisin eikä se toteuta minkään laista MVC:hen liittyvää pohjaa. Tämän vuoksi sitä käytetään hyväksi tutkiessa miten MVC tulisi toteuttaa web-sovelluskehykseen, jotta sen alkuperäiset määritelmät täyttyisivät \parencite{flask}.

\chapter{MVC:n vaatimukset}
Vaikka MVC:tä ei ole tarkoitettu alunperin web-sovelluksiin, voivat ne hyötyä MVC:n arkkitehtuurista. Suurin ongelma MVC:n käyttämisessä web-sovelluskehyksissä on palvelimen (server) ja asiakkaan (client) välinen ositus. Näkymä näytetään aina asiakkaan selaimessa sen omalla päätelaitteella HTML-kielellä. Malli ja Ohjain taas voivat olla ositettu teoriassa miten vain asiakkaan ja palvelimen välillä. Web-sovelluksissa kehittäjä pakotetaan osioimaan sovellus. MVC:n tulee olla riippumaton osioinnista. Osioinnin ei tule määrittää sovelluksen arkkitehtuuria \cite{ibm_watson}.

Pyramidin sovellusarkkitehtuuri on toteutettu käyttäen pohjana MVC"-arkkitehtuuria. Tämä ei kuitenkaan tarkoita sitä, että Pyramid toteuttaisi MVC:n teknisesti sellaisena kuin esimerkiksi Krasner määrittelee \cite{krasner_desc}. 
Tärkeimpänä vaatimuksena MVC:n toteutukselle on se, että näkymä ja ohjain luovat rajapinnan, jonka kautta käyttäjä keskustelee mallin kanssa. Malli ei saa olla suorassa yhteydessä käyttäjään \cite[s. 10]{reenskaug_tools}.
Mallin tulee olla riippumaton näkymästä ja ohjaimesta. Sen tulee myös hallita sovelluksen tilaa sekä pystyä antamaan informaatiota sovelluksen tilasta \cite{burbeck}. Näkymän on keskusteltava mallin kanssa sekä hoitaa mallilta saadun datan graafinen näyttäminen \cite[s.1]{reenskaug_orig}. Ohjain puolestaan ottaa vastaan syötteitä ja lähettää viestejä tämän perusteella näkymälle ja mallille \cite{burbeck}. Lisäksi Mallin tulee pystyä kommunikoimaan muutoksistaan ohjaimelle ja näkymälle. Näiden rajoituksien perusteella pystytään vastaamaan tutkielmassa siihen, toteutuuko MVC-arkkitehtuuri vai ei.


\chapter{MVC \& Pyramid}
Pyramidin kehittäjien dokumenteissa esitellään Pyramid MVC-kehyksenä, mutta samalla myös kyseenalaistetaan tämä väite. Erityisesti mallin ja ohjaimen määritelmä puuttuu \cite{Pyramid:intr}. Tässä
osiossa käsitellään Pyramidia MVC:n näkökulmasta. Tarkastelua varten toteutetaan Pyramidilla vastaava laskuri-sovellus kuin Krasnerin julkaisussa käyttäen Pyramidin tarjoamia ominaisuuksia \cite{krasner_desc}. Sovellus rajataan käyttämään SQL-tietokantaa mallin datan tallennukseen sekä \emph{URL dispatch} -tekniikkaa \cite{urldispatch, sql}. Sovelluksen pohjalta tarkastellaan MVC:n kannalta kolmea oleellisinta tiedostoa: \emph{views.py}, \emph{models.py} ja \emph{template.pt}. Näin tutkielman tarkastelu pystytään rajaamaan mahdollisimman pienelle alueelle, jolloin tutkielma on helpompi keskittää tarkastelemaan yksittäistä tekniikkaa.

Tässä osiossa tarkastellaan Pyramid-sovelluksen tiedostoja sekä niiden sisältöjä MVC-komponentteina. Erityisesti keskitytään ohjaimen ja näkymän toteukseen. Samalla tutkitaan voidaanko Pyramid havaintojen perusteella luokitella MVC:n toteuttavaksi sovelluskehykseksi.
Koska Pyramidissa MVC:n määrittely on hyvin epävakaalla pohjalla, tulee tehdä selvästi mitä ominaisuuksia komponenteilta vaaditaan. Esimerkiksi \emph{views.py} ja \emph{template.pt} -tiedostot ovat tarkoitettu toimimaan yhdessä näkymänä, jolloin ohjaimen toteuttamat tehtävät sisällytettäisiin näkymään. MVC:tä tutkiessa täytyy sovelluksen komponentit kuitenkin jakaa kolmeen osaan.  Tiedosto \emph{views.py}
sisältää paljon ohjaimelle yhteisiä piirteitä, joten se erottuu selvästi \emph{template.pt} -tiedostosta. Tutkielmassa oletetaan, että \emph{models.py} sisältää mallin, \emph{views.py} ohjaimen ja \emph{template.pt} näkymän.

\section{Tiedostojen tarkastelu}
\lstset{
  literate={ö}{{\"o}}1
           {ä}{{\"a}}1,
xleftmargin=30pt,
escapeinside={/*@}{@*/},
backgroundcolor=\color{light-gray}
}
\lstset{numbers=left}
\begin{lstlisting}[language=Python]
# Tiedosto: models.py
class Counter(Base):
    __tablename__ = 'counter'

    # Asetetaan mallille atribuutit, jotka
    # vastaavat mallin tilasta.
    id = Column(Integer, primary_key=True)
    name = Column(Unicode(255), unique=True)
    value = Column(Integer)

    # Määritellään luokalle konstruktori,
    # joka saa parametreiksi nimen ja alkuarvon.
    def __init__(self, name, value):
        self.name = name
        self.value = value

    def increment(self):
        self.value += 1

    def decrement(self):
        self.value -= 1

# Luodaan instanssi mallista ja rekisteröidään 
# se sovellukseen.
def populate():
    session = DBSession()
    model = Counter(name=u'counter', value=0)
    session.add(model)
\end{lstlisting}
\emph{Models.py} -tiedosto sisältää malliin liittyvän ohjelmakoodin. Jokaiselle mallille luodaan aina oma luokkansa, jossa määritellään mallin ominaisuudet. Malli rekisteröidään \emph{populate} -funktion kautta, jota kutsutaan Pyramidin toimesta.
Ohjelmakoodista nähdään, että malli ei luo minkäänlaista riippuvuutta näkymään tai ohjaimeen. Se myös pitää huolen datan käsittelystä.  Malli on siis Pyramidissa itsenäinen komponentti, joka huolehtii sovelluksen tilasta. Tämän perusteella malli toteutuu Pyramidissa MVC-arkkitehtuurin mukaisesti.
\emph{Views.py} -tiedostossa määritellään näkymän ohjelmakoodi. Pyramidissa on konkreettisesti määritelty ohjelmakooditasolla vain malli ja näkymä.
Jokaista näkymää kohden on oma funktio, joka ottaa vastaan \emph{request}-olion. Request-oliossa tuodaan sovellukselle kaikki tieto käyttäjästä ja sovelluksen viesteistä. Tämän perusteella tulkitaan request-oliossa tuotu data käyttäjän syötteiksi.
Vaikka \emph{views.py} nimetään Pyramidissa näkymäksi, on se toteutukseltaan hyvin lähellä ohjainta. Tästä syystä tarkastellaan funktion toteutusta mahdollisena ohjaimena. Tästä eteenpäin puhutaessa ohjaimesta Pyramidissa, tarkoitetaan sillä \emph{views.py} -tiedoston sisältämää funktiota.

\lstset{
  literate={ö}{{\"o}}1
           {ä}{{\"a}}1,
xleftmargin=30pt,
escapeinside={/*@}{@*/},
}
\begin{lstlisting}[language=Python]
# Tiedosto: views.py
@view_config(route_name='counter_view',
	renderer='templates/counter.pt')
def counter_view(request):
    dbsession = DBSession()

    # Rekisteröidään malli.
    counter = dbsession.query(Counter).filter(
	Counter.name==u'counter').first()
    try:
        request.params['minus'].
        counter.decrement()
    except KeyError:
        pass

    try:
        request.params['plus']
        counter.increment()
    except KeyError:
        pass

    # Palautetaan laskurin arvo, joka tulkitaan
    # ja näytetään template.pt -tiedostssa
    return {'value': counter.value}
\end{lstlisting}
Määritellään funktiolle URL-osoite sekä liitetään siihen template-tiedosto (2). Tämän jälkeen rekisteröidään malli mukaan funktioon (8). Koska tarkastelemme funktiota ohjaimena, voimme tulkita templaten näkymäksi. Tällöin funktioon rekisteröidään malli sekä näkymä, jolloin rekisteröinnin puolesta se toteuttaa ohjaimelle tarkoitetut ominaisuudet MVC:ssä. Funktioon lisätään myös toiminto laskurin vähentämiselle. Mallin \emph{value} -arvoa muutetaan, kun request-oliosta löytyy tietty parametri. Funktio palauttaa paluuarvona mallin arvon, joka tuodaan käsiteltäväksi templateen. Funktio siis ottaa vastaan syötteitä ja niiden perusteella lähettää viestejä
mallille sekä näkymälle. Tämän perusteella todetaan, että se täyttää rajauksessa määrätyt ohjaimen ominaisuudet. 

Templatessa yhdistetään HTML-merkkauskieli ja sovelluksen ohjelmakoodi. Tästä generoidaan HTML-sivu, joka näytetään selaimelle. Koska malli sekä ohjain on jo määritelty, täytyy selvittää täyttääkö template näkymälle määritellyt ominaisuudet.
\begin{lstlisting}[language=Python]
  <body>
    <h1>${value}</h1>
    <form action="./" method="get">
      <button type="submit" name="plus" value="plus">
		Increment </button>
      <button type="submit" name="minus" value="minus"> 
		Decrement </button>
    </form>
  </body>
\end{lstlisting}
Templatessa luodaan lomake kahdelle painikkeelle, joista kumpikin lähettää lomakkeen eteenpäin \emph{counter\_view} -funktiolle. Lomakkeen tiedot tulevat funktiolle request-oliossa, joka sisältää tässä tapauksessa \emph{plus}- tai \emph{minus}-parametrin riippuen siitä kumpaa painiketta on painettu. Lomakkeen tiedot lähetetään samaan osoitteeseen (3), mistä sitä on alunperin kutsuttukin. Eron tuo kuitenkin request-oliossa tuodut parametrit.  Otsikossa (2) tuodaan näkyviin laskurin sen hetkinen arvo, joka saadaan tietoon ohjaimelta.

Templatessa hoidetaan sovelluksen graafinen puoli, joten se vastaa ominaisuuksiltaan näkymää. \\ Ongelmaksi muodostuu kuitenkin näkymän ja mallin välinen kommunikointi. Näkymä ei ole yhteydessä malliin suoraan, vaan tarvitsee ohjaimen kautta tiedon mallin tilasta.  Näkymä ei siis sellaisenaan toteuta sille asetettuja ominaisuuksia. \\

\section{Sovelluksen toiminta}
Alla olevassa kuvassa esitellään visuaalisesti miten laskurisovellus muodostaa HTML-sivun, kun käyttäjä painaa sivulla \emph{Increment} -painiketta. Kuvan
vaiheet toteutetaan numerojärjestyksessä alkaen ensimmäisestä.
\begin{figure}[h]
\centering
\includegraphics[scale=0.87]{laskurisovellus.png}
\caption{Laskurisovelluksen toiminta}
\end{figure}
\lstset{
  literate={ö}{{\"o}}1
           {ä}{{\"a}}1,
xleftmargin=30pt,
escapeinside={/*@}{@*/},
}

\begin{lstlisting}
/*@Palvelimelta pyydetään HTTP-protokollan mukaisesti sivua \emph{plus} -parametrilla.@*/
/*@Palvelin pyytää sovellukselta sivua. Parametri tuodaan sovellukselle \emph{request}-oliossa.@*/
/*@Ohjain käskee mallia muuttamaan tilaansa ja pyytämään samalla tiedon muutoksen jälkeisestä arvosta.@*/
/*@Ohjain palauttaa mallin arvon, joka käsitellään templatessa.@*/
/*@Templatessa generoidaan HTML-sivu, joka tuodaan palvelimelle vastauksena.@*/
\end{lstlisting}

\section{Yhteenveto}
Malli sekä ohjain toteutuvat Pyramidissa MVC:n mukaisesti yksittäisinä komponentteina, mutta kommunikaatio näiden välillä ei mene MVC:n mukaisesti. Malli on itsenäinen komponentti, jolla ei ole riippuvuutta näkymään tai ohjaimeen. Se myös huolehtii sovelluksen käsittelemästä datasta ja vastaa tarvittaviin pyyntöihin. Se ei kuitenkaan pysty kertomaan ohjaimelle ja näkymälle omista muutoksistaan, koska kaikki muutokset tulevat näkymälle asti vastauksena vasta HTTP-pyynnön mukana. Ohjain taas huolehtii request-oliossa tulevista syötteistä ja vaikuttaa malliin sekä näkymään. Näkymä toteuttaa sovelluksen graafisen näyttämisen selaimelle, mutta ei toteuta sille määrättyjä sääntöjä. MVC:ssä näkymän tarkoitus on kommunikoida suoraan mallin kanssa. Tämä ei kuitenkaan onnistu Pyramidissa, jossa näkymä on yhteydessä vain ohjaimeen. Pyramidin MVC-toteutusta tarkastellessa tulee ottaa huomioon yksittäisten komponenttien toteutus sekä niiden välinen yhteistyö. Yksittäiset komponentit toteutuvat Pyramidissa MVC-arkkitehtuurin mukaisesti, mutta niiden välinen yhteistyö ei toteudu .Pyramidin MVC-komponenttien kommunikointi voidaan esittää käyttäen pohjana muokaten Krasnerin kommunikaatiomallia \cite{krasner_desc}.
Alla esitellyssä kuvassa havainnollisestetaan, kuinka mallin ja näkymän kommunikaatio puuttuu täysin ja kaikki data tuodaan ohjaimen kautta. Lisäksi Malli ei pysty kommunikoimaan muutoksista suoraan ohjaimelle ja näkymälle vaan tarvitsee aina HTTP-pyynnön.

\begin{figure}[h]
\centering/
\includegraphics[scale=0.85]{pyramid_mvc.jpg}
\caption{Pyramidin kommunikointi MVC-komponenttien kesken. Kuva on muokattu Krasnerin esittelemästä kommunikointimallista \cite{krasner_desc} }
\end{figure}

Yllä esitetyssä kommunikaatiomallissa ohjain saa vastaan request-olion, jossa tuodaan kaikki tarvittava tieto käyttäjästä. Tämän perusteella ohjain ohjaa mallia
sekä muuttujaa. Samalla se pyytää mallilta tietoa sovelluksen tilasta ja välittää tiedon näkymälle. Näkymä taas välittää ohjaimen tuoman datan käyttäjälle graafisena.


Alunperin Pyramidin dokumentaatiossa kyseenalaistettiin mallin sekä ohjaimen toteutus \cite{Pyramid:intr}. Tutkimuksen pohjalta voidaan kuitenkin
todeta, että ongelmaksi ei muodostu yksittäisien komponenttien toteutus, vaan komponenttien välinen kommunikointi. Erityisesti näkymän ja mallin yhteistyö jää
kokonaan puuttumaan, jolloin saadaan ristiriita MVC:n alkuperäisen määritelmän kanssa \cite[s. 1]{reenskaug_orig}. Tämä johtuu siitä, että Pyramidissa \emph{views.py} -tiedoston sisältämä ohjelmakoodi on nimensä mukaisesti tarkoitettu näkymäksi ja
\emph{template.pt} -tiedosto katsotaan osaksi samaa komponenttia. Tutkimuksen tuloksien perusteella voidaan kuitenkin todeta, että \emph{views.py} -tiedoston näkymä-funktio toteuttaa kaikki
ohjaimelle määritellyt ominaisuudet. Template puolestaan hoitaa sovelluksen graafisen puolen, joten sen ominaisuudet ovat mahdollisimman lähellä näkymää. Template ei kuitenkaan riitä toteuttamaan näkymän ominaisuuksia, koska se on täysin riippuvainen ohjaimesta. Lisäksi Malli ei pysty kommunikoimaan suoraan näkymälle ja ohjaimelle ilman, että ohjain/näkymä pyytää Mallilta nykyistä tilaa.


\chapter{MVC:n toteutus Flask-sovelluskehyksellä}

\chapter{Tulokset}
\printbibliography
\appendix



\end{document}
